\section{Определители. Свойства.}

Определитель (детерминант) --- скалярная величина (число) характеризующая квадратную матрицу.

Она нужна, например, для получения обратной матрицы или для понятия обратима ли матрица, при решение СЛАУ и для много чего ещё, что \href{https://math.stackexchange.com/a/160792}{мы даже не прошли}.

Перестановка называется {\bf четной}, если она содержит четное число инверсий элементов. Нечетная перестановка содержит нечетное число инверсий.  

{\bf Инверсия} --- пара элементов, которые находятся вне их естественного порядка. ($i < j \land a_i > a_j$)

$(j_1, j_2, \dots, j_n)$ перестановка чисел от $1$ до $n$

$$\displaystyle det(A) = \sum_{(j_1, j_2, \dots, j_n)} (-1) ^ {N(j_1, j_2, \dots, j_n)} \cdot a_{1, j_1}  \cdot a_{2, j_2} \cdot \dots \cdot  a_{n, j_n}  = \sum_{(j_1, j_2, \dots, j_n)} \left( (-1) ^ {N(j_1, j_2, \dots, j_n)} \cdot \prod_{i=1}^{i\le n} a_{i, j_i} \right)$$

Где $N(j_1, j_2, \dots, j_n)$ --- количество инверсий перестановки. (в лекции говорили что берём $+$ если чётная и $-$ если не чётная, что собственно одно  и тоже)

Говоря человеческим языком, по всем перестановкам номеров столбцов берём сумму от произведения чётности перестановки помноженную на  произведение всех элементов матрицы с номерами $a_{i, j_i}$ (номер столбца выбираем из перестановки под номером равным номеру строки).


Вычисление определителя которое все использует, так как предыдущее требует порядка $\O(n! \cdot n)$ операций.

{\bf Формула разложением по строке}:

$A =
\begin{pmatrix} a_{11}
\end{pmatrix} = a_{11}$

$A =
\begin{pmatrix} a_{11} & a_{12} 
	\\a_{21} & a_{22}
\end{pmatrix} = a_{11} \cdot a_{22} - a_{21} \cdot a_{12}$

$ A =
\begin{pmatrix} a_{11} & a_{12} & \cdots & a_{1n}
	\\a_{21} & a_{22} & \cdots & a_{2n}
	\\ \vdots & \vdots & \ddots & \vdots
	\\ a_{n1} & a_{n2} & \cdots & a_{nn}
\end{pmatrix}$

$$\displaystyle det(A) = \sum _{j=1}^{n}(-1)^{1+j}a_{1, j} {M}_{1, j}$$

$M_{i, j}$ минор элемента $a_{i, j}$ --- это определитель полученный вычёркиванием $i$-ой строки и $j$-го столбца матрицы $A$ при сохранение порядка остальных элементов.

Для фиксированного $i$ (строки) $\displaystyle det(A) = \sum _{j=1}^{n}(-1)^{i+j}a_{1, j} {M}_{i, j}$

Свойства с доказательствами:

\begin{enumerate}
	\item Если в матрице есть нулевая строка, то определитель равен $0$.
	
	Посчитаем определитель по строке там где находятся нули, и тогда у нас получается сумма $0$ по формуле. 
	
	\item $det(A) = det(A^{T})$
	
	Докажем по индукции: пусть $det(A) = det(A^T)$ для матриц порядка $n - 1$ --- верно.
	
	Докажем для матрицы порядка $n$:
	
	Помогите, я реально не понимаю, что за логика в вики написанная. 
	
	\item Перестановка строк местами меняет только знак определителя.
	
	Для $n=1$ не имеет смысла, далее докажем по индукции: База $n=2$ выполняется по определению, пусть мы доказали для $n-1$ выполняется утверждение. Пусть $i, j$ номера строк, тогда $B$ матрица полученная из $A$ если поменять строки $i$ и $j$ местами, а $k$ номер строки отличный от $i$ и $j$. Тогда распишем определитель по строке $k$ : $det(B) = (-1) ^ {k + 1} b_{k, 1} \cdot det(B_{k, 1}) + (-1) ^ {k + 2} b_{k, 2} \cdot det(B_{k, 2}) + \dots + (-1) ^ {k + n} b_{k, n} \cdot det(B_{k, n}) = $
	
	Воспользуемся двумя простыми утверждениями:
	
	\begin{itemize}
		\item $b_{k, z} = a_{k, z}$ так как мы не изменяли данную строку.
		\item $det(B_{k, z})$ это определитель матрицы размера $(n - 1)$ на $(n - 1)$,  содержащий строки $i$ и $j$, по индукции уже знаем, что $det(B_{k, z}) = -det(A_{k, z})$. 
	\end{itemize}

	Получаем, что $= (-1) ^ {k + 1} b_{k, 1} \cdot -det(A_{k, 1}) + (-1) ^ {k + 2} b_{k, 2} \cdot -det(A_{k, 2}) + \dots + (-1) ^ {k + n} b_{k, n} \cdot -det(A_{k, n}) = \\ = -det(A)$.
	
	\item Определитель матрицы, которая содержит две одинаковые строки, равен $0$.
	
	Поменяем две равные строки. Тогда $A$ и полученная $A'$ не различимы, а следовательно по определению $det(A) = det(A')$. По свойству 3 получим, что $det(A) = -det(A')$. Итого получаем, что $det(A) = -det(A)= 0$.
		
	\item Вынос константы у строки
	
	Распишем определитель по данной строке, вынесем константу за скобку. 
		
	\item Определитель матрицы, которая содержит две пропорциональные строки, равен $0$. 
	
	По свойству 5 выносим константу у одной из этих двух строк, чтобы получить равные строки, и по свойству 4 получаем что определитель равен $c \times 0 = 0$.
	
	\item * не было на лекции *, но очень важное
	
	Если к строке $k$ добавить последовательность  чисел $b$ длины $n$, то определитель будет равен сумме определителей исходной матрицы и матрицы у которой заменили $k$-ую строку на последовательность чисел.
	
	Расписывание определителя по $k$-ой строке.	
	
	$A''$ матрица полученная из $A$ прибавлением к строке $k$ последовательности $b$.
	
	\begin{equation}
		\begin{split}
		\displaystyle det(A'') &= \sum _{j=1}^{n}(-1)^{k+j} (a_{k, j} + b_{j}) {M}_{k, j} = \\ 
		&= \sum _{j=1}^{n}(-1)^{k+j} a_{k, j} {M}_{k, j} + \sum _{j=1}^{n}(-1)^{k+j} b_{j} {M}_{k, j} = \\
		&= det(A) + det(A')
		\end{split}	
	\end{equation}
	Где $A'$ --- в матрице $A$ заменили $k$-ую строку на последовательность $b$.

	\item * не было на лекции *, но очень важное
	
	Определитель не изменится, если к элементам любой его строки прибавить соответствующие элементы другой строки.
	
	Пусть $A'$ матрица полученная из $A$ путём замены данной строки на другую строку.
	
	Пусть $A''$ матрица полученная из $A$ складывания одной строки с другой.
	
	По свойству 4 получаем, что $det(A') = 0$
	
	Из свойства 7 следует, что $det(A'') = det(A) + det(A') = det(A)$.
	
	Комбинируя свойства выше доказывается, что можно добавлять не только одну стоку один раз, но и любое произвольное количество раз, так как по свойству $6$ в доказательстве выше $det(A') = 0$.
	
	\item Если есть строка, которая является линейной комбинацией других двух строк, то определитель равен $0$.
	
	$l_z = \alpha l_i + \sigma l_j$
	
	Из следствия свойства 8 получается что определитель такой матрицы будет таким-же как и определитель матрицы, в которой строка $z$ равна $\alpha l_i$ + по свойству 4 получаем что определитель такой матрицы равен $0$.
	
	\item $det(A \cdot B) = det(A) \cdot det(B)$
	
	Это мы точно не умеем доказывать((( TODO напишите если умеете)
	
	\href{https://math.stackexchange.com/questions/60284/how-to-show-that-detab-deta-detb}{link1}
	
	\href{https://www.proofwiki.org/wiki/Determinant_of_Matrix_Product}{link2}
	
\end{enumerate}


\pagebreak
