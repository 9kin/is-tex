\section{Поле комплексных чисел.}

* В других билетах особо не спрашивают, что такое группа, кольцо и поле. Поэтому я решил добавить это в данный билет.

{\bf Множество} --- совокупность объектов с общим свойством.

Существуют бинарные и унарные операции над множествами. Бинарная принимает два аргумента и возвращает один результат, унарная аналогично принимает один аргумент и возвращает один результат.

\down

{\bf Алгебраическая операция} на множестве $Х$ --- соответствие, при котором каждой паре элементов из множества $Х$ соответствует единственный элемент этого же множества.


% Источник : http://old.math.nsc.ru/~vasand/Lectures/algebra_small.pdf стр 14
% http://math.phys.msu.ru/data/86/LECTURES(2016)1.pdf

{\bf Алгебраическая структура} (\href{https://ru.wikipedia.org/wiki/%D0%90%D0%BB%D0%B3%D0%B5%D0%B1%D1%80%D0%B0%D0%B8%D1%87%D0%B5%D1%81%D0%BA%D0%B0%D1%8F_%D1%81%D0%B8%D1%81%D1%82%D0%B5%D0%BC%D0%B0}{Алгебраическая система в википедии}) --- называется объект, являющийся совокупностью непустого множества $A$и непустого набора {\bf алгебраических операций} $f_1, f_2, \dots, f_k, \dots$.

 Из определения следует, что множество $N$ с операцией $-$ не является алгебраической структурой, так как $-$ на таком множестве не замкнутая операция.

\down

Алгебраические структуры далее классифицируются так:

\begin{itemize}
	\item Группой $(G, \oplus)$ называется алгебраическая система с одной бинарной 
	{\bf алгебраической операцией}
	 
	\begin{enumerate}
		\item Определена бинарная операция. 
		
		Например, сложение или умножение, поэтому обычно говорят, что "группа по сложению" или "группа по умножению"
		
		\item Операция является ассоциативной.
		
		\item Существует нейтральный элемент для данной операции в данном множестве.
		
		Для сложение $\mathbb{0}$, так как $\forall a \in G \, \mathbb{0} + a = a \in G$.
		
		Для умножения $\mathbb{1}$, так как $\forall a \in G \, \mathbb{1} \cdot a = a \in G$.
		
		\item Каждый элемент множества имеет обратный. Для структуры по сложению: $\forall a \in G \, \exists a' \in G \, a' + a = \mathbb{0}$
	
	\end{enumerate}
	
	Группа называется {\bf абелевой}, если операция в ней коммутативна, т.е. $$\forall x, y \in G \, \, x \oplus y = y \oplus x$$

	\item {\bf Кольцом} называется непустое множество $R$ с двумя заданными на нём бинарными операциями $+$ (сложение) и $\cdot$ (умножение), которые обладают следующими свойствами:
	
	* Разумеется $+$ и $\cdot$ это просто обозначения некоторых операций, что-бы не писать операция $1$.
		
	\begin{enumerate}
		
		\item  относительно сложения $+$ множество $R$ образует {\bf абелеву} группу, называемую аддитивной группой кольца; нейтральный элемент этой группы называется нулём и обозначается $\mathbb{0}$
		
		\item умножение $\cdot$ дистрибутивно относительно сложения: для любых $a, b, c \in R$ имеют место соотношения
		
		$$(a + b) \cdot c = a \cdot c + b \cdot c, c \cdot (a + b) = c \cdot a + c \cdot b$$
		
	\end{enumerate}
	
	Если операция умножения коммутативна (ассоциативна), то кольцо называется коммутативным (ассоциативным).
	
	\item {\bf Полем} называется коммутативное ассоциативное кольцо с единицей, в котором любой ненулевой элемент обратим.
	
\end{itemize}

\down

\down

\begin{center}
{\bf Поле комплексных чисел}
\end{center}

{\bf Поле комплексных чисел} --- множеств упорядоченных пар $\mathbb{C} = \{(x, y) | x, y \in \mathbb{R}\}$


Пусть $z_1 = (x_1, y_1), z_2 = (x_2, y_2)$.

\begin{itemize}

\item Введём $+$: 

$$z_1 + z_2 = z_2 + z_1 = (x_1 + x_2, y_1 + y_2)$$

\item Введём $\cdot$:

$$z_1 \cdot z_2  = z_2 \cdot z_1  = (x_1 \cdot x_2 - y_1 \cdot y_2, x_1 \cdot y_2 + x_2 \cdot y_1)$$

\item Пусть $$\mathbb{0}_{\mathbb{C}} = (0, 0), \mathbb{1}_{\mathbb{C}} = (1, 0)$$

\end{itemize}

{\bf Множество $(\mathbb{C},+,\cdot)$ является полем.}

Пусть $$i = (0, 1)$$

Тогда любое комплексное число можно обозначать так $z = (a, b) == a + ib$

$$i ^ 2 = (0, 1) \cdot (0, 1) = \dots = -1$$

Отсюда получаем $i = \sqrt{-1}$ --- мнимая единица.



\pagebreak